\documentclass[11pt]{article}
\usepackage[spanish]{babel}
\usepackage[utf8]{inputenc}
\usepackage[margin=1.5in]{geometry}

\begin{document}
\title{Estructuras de Datos\\ \large Laboratorio 7}
\author{Vicente Varas Pavez}
\maketitle
\section *{Ejercicio 3}
Se debe a que al insertar los elementos de forma ordenada, el árbol que se genera tiene forma de lista. Esto significa que la búsqueda implementada se comporta como una búsqueda iterativa, de orden O(n) en tiempo, mientras que para un árbol balanceado la busqueda será de orden O(log(n)).
\section *{Ejercicio 4}
Que tan balanceado esté un árbol tiene que ver con la relación entre su número de nodos y su altura. En un árbol completamente desbalanceado, como el que se produce al insertar  los números de forma ordenada, la altura es n, mientras que en uno balanceado es log(n). Según lo visto en clases, un árbol binario se considera balanceado si todos los niveles sobre el último están llenos.
\end{document}\grid
\grid
